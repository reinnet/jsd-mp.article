In this section, we review the works that have been done on service chain deployment and resource assignment in NFV, and mainly we focus on the works that consider management requirements of chains.

In NFV, resource assignment is composed of
allocating computing and memory resources for virtual machines to run the chains' virtual network functions and assigning network resources to the chains' virtual links. Various objective functions, e.g., minimizing cost, maximizing profit and minimizing power consumption can be aimed in the resource management problem where a wide range of additional, e.g., end-to-end delay, fault tolerance as well as management requirements need to be satisfied.

\subsection{Resource Allocation in NFV}
The VNF placement problem has received substantial attention in the literature \cite{GilHerrera2016}.
We will concentrate on network function chain placement which dynamically steers
traffic through an ordered list of Service Functions from 3 categories that are discussed in \cite{Laghrissi2019}.
Objectives like energy consumption minization, cost optimization, Quality of Service (QoS), resource usage, reliablity,
and load balancing.
Authors consider this problem statically (off-line) or dynamically (on-line) and from the aformentioned objective they sometimes
consider more than one objective. Solutions vary from heuristics to meta-heuristic like simulated annealing.
For getting more information on these placement problems please refer to \cite{Laghrissi2019}.

\subsection{Management Resources in NFV}
As already mentioned, this work considers management resources. To best of our knowledge, work \cite{AbuLebdeh2017}, its next work \cite{AbuLebdeh20172}, and \cite{Chiang2019} are the only ones that consider VNFM and other management resources in SFC deployment.

In \cite{AbuLebdeh2017}, the authors studied the problem of VNFM placement in a distributed NFV infrastructure under the assumption that chains have already been deployed and consequently, the location of VNFs are known.
The objective function of the VNFM placement problem is to minimize the operational costs which is:
\begin{itemize}
    \item Life cycle Management Cost
    \item Compute Resources Cost
    \item Migration Cost
    \item Reassignment Cost
\end{itemize}
Delay constraint on management links is also considered.
The authors used tabu search algorithm because of its superior results to other techniques in FLP, for finding a polynomial solution. They start from a feasible placement and each step they improve VNFM placement by doing one of these moves:
\begin{itemize}
    \item Reassignment
    \item Relocation
    \item Bulk
    \item Deactivation
\end{itemize}
While this is the first work that investigated the management resource assignment in NFV, it does not the joint SFC deployment and VNFM placement.

In \cite{AbuLebdeh20172} authors solve VNFO placement problem that is far from our current work that doesn't consider VNFO.
Authors consider the same system as \cite{AbuLebdeh2017} but here they want to place VNFO and VNFM jointly, trying to minimize operational cost as defined by \cite{AbuLebdeh2017}.
They propose a two step placement algorithm that first place VNFOs and then place VNFMs. Each of these steps use Tabu-Search method.

In \cite{Chiang2019} authors consider autonomy for VNFMs that selects their managed VNFs dynamically and use game theory to achieve a distributed solution to the VNFM Placement Problem as desribed in \cite{AbuLebdeh2017}.
Authors consider the same system model from \cite{AbuLebdeh2017} and try minimize operational cost consists of bandwidth cost and compute cost.
In our work relation between VNFMs and VNFs are static and VNFM cannot change these relation by their own.

To conclude, the problem of joint SFC deployment and VNFMs placement has not yet been consider. In the rest of this paper, we formulate the problem and propose solution for it.

\subsection{Current Work}

The closest work to current research is \cite{AbuLebdeh2017} but it assumes that the VNFs' placement is known and solves only the VNF manager placement problem. Current research also considers more constraints that \cite{AbuLebdeh2017} on VNFM placements that we discuss deeply in the following sections and summarize them here:

\begin{itemize}
    \item Current research considers license cost for VNFM instances
    \item In current research each physical node has its specific list of nodes that can run its VNFM. This constraint provides a great flexibility for implementing management policies.
\end{itemize}

At the end current research and \cite{AbuLebdeh2017} have different objectives but we will try to compare their results and solutions.
