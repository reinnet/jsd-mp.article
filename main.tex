\documentclass[3p]{elsarticle}

\usepackage{lineno,hyperref}
\modulolinenumbers[5]

\journal{Journal of \LaTeX\ Templates}

\bibliographystyle{elsarticle-num}

\begin{document}

\begin{frontmatter}

\title{Virtualized Network Service Function Chaining Subject to Management Resource Constraint}

\author[aut]{Parham Alvani \corref{correspondingauthor}}
\ead{parham.alvani@aut.ac.ir}
\author[aut]{Bahador Bakhshi}
\ead{bbakhshi@aut.ac.ir}

\cortext[correspondingauthor]{Corresponding author.}

\address[aut]{Amirkabir University of Technology, Tehran, Iran}

\begin{abstract}
In the old times, Network providers use hardware network functions to create their service chains, but a change in this manner is difficult and may cause many service distribution.
SFC and NFV is the solution to this difficulty. By using SFC and NFV, providers can provision chains dynamically and then change them in runtime.
One of the main requirements is management and monitoring for the chains.
In this research, we consider the chain acceptance problem subject to management resources. In the first step, we formulate
problem with ILP and then implement it in CPLEX framework. As we know, ILP problems are NP-Hard, so we need a Polynomial-Time solution to the problem.
In this research, we create a heuristic algorithm and compare its result with the optimal solution. In the end, the heuristic solution produces near-optimal results in the polynomial time.
\end{abstract}

\begin{keyword}
\end{keyword}

\end{frontmatter}

\section{Related Works}

Here we want to review on works that have been done on Service Placement and Resource Assignment
in SFC and NFV in these works we emphasize management resources and service placement.
In this work, we see resources assignment as assigning the network, computation and memory resources.
We assign computation and memory resources to virtual machines to run the services and assign network resources to links for running the virtual links of chains.

\subsection{Resource Allocation in NFV}
The VNF placement problem has received substantial attention in the literature.
Formally the problem is defined as selecting locations for a chain of VNF instances.
There are multiple objectives to consider with VNF placement.
For Example, In [6] authors want to accept the maximum number of chains by considering only the processing and network resources and types for VNFs to create instances from these types.

\subsection{Management Resources in NFV}
In the NFV literature, each NFV chain must be monitored and managed by VNFM
These types of resources consider management resources in this work.
To best of our knowledge \cite{AbuLebdeh2017} and their next work [?] are the only ones that consider VNFM and other management resources in the placement process.
In [2] authors try to optimize VNFM placement over a distributed NFVI

Our work considers these problems as a joint problem and wants to place chains and their VNFM at the same time. To best of our knowledge, this work hasn't been done in the literature. By considering the joint problem you may not accept a chain that you don't have any management resource for it or you can accept many chains that have little management resource requirement.


\bibliography{references}

\end{document}
